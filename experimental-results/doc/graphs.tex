\documentclass[runningheads]{llncs}

%%
%% Corresponding author:
%% Randal E. Bryant (Randy.Bryant@cs.cmu.edu)
%% 

%\usepackage{latexsym}
\usepackage{times}
\usepackage{amsmath}
\usepackage{amssymb}
%\usepackage{alltt}
\usepackage{stmaryrd}
\usepackage{graphicx}
\usepackage{tikz}
\usepackage{pgfplots}
\usepackage{cite}
\usepackage{booktabs}
\usepackage{adjustbox}
\usepackage{marvosym}
\usepackage{ifthen}
\usepackage{fancyvrb}
\usepackage[many]{tcolorbox}



\bibliographystyle{splncs04}

%% Change tt font
%usepackage{DejaVuSansMono}
\usepackage{inconsolata}

\usepackage[T1]{fontenc}


\newboolean{longv}
\setboolean{longv}{false}


\definecolor{redorange}{rgb}{0.878431, 0.235294, 0.192157}
\definecolor{lightblue}{rgb}{0.95, 0.9, 0.99}
\definecolor{clearyellow}{rgb}{0.964706, 0.745098, 0}
\definecolor{clearorange}{rgb}{0.917647, 0.462745, 0}
\definecolor{mildgray}{rgb}{0.54902, 0.509804, 0.47451}
\definecolor{softblue}{rgb}{0.643137, 0.858824, 0.909804}
\definecolor{bluegray}{rgb}{0.141176, 0.313725, 0.603922}
\definecolor{lightgreen}{rgb}{0.9, 0.99, 0.9}
\definecolor{redpurple}{rgb}{0.835294, 0, 0.196078}
\definecolor{midblue}{rgb}{0, 0.592157, 0.662745}
\definecolor{clearpurple}{rgb}{0.67451, 0.0784314, 0.352941}
\definecolor{browngreen}{rgb}{0.333333, 0.313725, 0.145098}
\definecolor{darkestpurple}{rgb}{0.396078, 0.113725, 0.196078}
\definecolor{greypurple}{rgb}{0.294118, 0.219608, 0.298039}
\definecolor{darkturquoise}{rgb}{0, 0.239216, 0.298039}
\definecolor{darkbrown}{rgb}{0.305882, 0.211765, 0.160784}
\definecolor{midgreen}{rgb}{0.560784, 0.6, 0.243137}
\definecolor{darkred}{rgb}{0.576471, 0.152941, 0.172549}
\definecolor{darkpurple}{rgb}{0.313725, 0.027451, 0.470588}
\definecolor{darkestblue}{rgb}{0, 0.156863, 0.333333}
\definecolor{lightpurple}{rgb}{0.776471, 0.690196, 0.737255}
\definecolor{softgreen}{rgb}{0.733333, 0.772549, 0.572549}
\definecolor{medgreen}{rgb}{0.34, 0.65, 0.34}
\definecolor{offwhite}{rgb}{0.839216, 0.823529, 0.768627}

\newcommand{\one}{\mbox{\bf 1}}
\newcommand{\zero}{\mbox{\bf 0}}
\newcommand{\leafone}{\top}
\newcommand{\leafzero}{\bot}
\newcommand{\booland}{\land}
\newcommand{\boolor}{\lor}
\newcommand{\boolxor}{\oplus}
\newcommand{\boolnot}{\neg}
\newcommand{\tautology}{\top}
\newcommand{\nil}{\bot}
\renewcommand{\obar}[1]{\overline{#1}}
\newcommand{\ite}{\mbox{\it ITE}}

\newcommand{\opname}[1]{\mbox{\sc #1}}
\newcommand{\andop}{\opname{And}}
\newcommand{\orop}{\opname{Or}}
\newcommand{\notop}{\opname{Not}}
\newcommand{\implyop}{\opname{Imply}}
\newcommand{\simplifyop}{\opname{Simplify}}
\newcommand{\constrainop}{\opname{Constrain}}
\newcommand{\applyop}{\opname{Apply}}
\newcommand{\applyand}{\opname{ApplyAnd}}
\newcommand{\applyor}{\opname{ApplyOr}}
\newcommand{\applyimply}{\opname{ProveImplication}}
\newcommand{\applyandimply}{\opname{ApplyAndProveImplication}}



\newcommand{\func}[1]{\llbracket#1\rrbracket}

\newcommand{\turnstile}{\vdash}
\newcommand{\fname}[1]{\mbox{\small\sf #1}}

\newcommand{\lo}{\fname{Lo}}
\newcommand{\hi}{\fname{Hi}}
\newcommand{\var}{\fname{Var}}
\newcommand{\val}{\fname{Val}}
\newcommand{\stephd}{\fname{HD}}
\newcommand{\stephu}{\fname{HU}}
\newcommand{\stepld}{\fname{LD}}
\newcommand{\steplu}{\fname{LU}}

\newcommand{\hsc}[1]{{\MakeUppercase{#1}}}


\newcommand{\rupbdd}{${\sf RUP}_{\sf  BDD}$}
\newcommand{\rup}{{\sf RUP}}
\newcommand{\drat}{{\sf DRAT}}


\newcommand{\pgbdd}{{\sffamily\scshape pgbdd}}
\newcommand{\Pgbdd}{{\sffamily\scshape  Pgbdd}}
\newcommand{\pgpbs}{{\sffamily\scshape  pgpbs}}
\newcommand{\Pgpbs}{{\sffamily\scshape  Pgpbs}}
\newcommand{\kissat}{{\sffamily\scshape kissat}}
\newcommand{\Kissat}{{\sffamily\scshape  Kissat}}
\newcommand{\lingeling}{{\sffamily\scshape lingeling}}
\newcommand{\Lingeling}{{\sffamily\scshape  Lingeling}}

%\newcommand{\pgbdd}{\textsc{pgbdd}}
%\newcommand{\Pgbdd}{\textsc{Pgbdd}}
%\newcommand{\pgpbs}{\textsc{pgpbs}}
%\newcommand{\Pgpbs}{\textsc{Pgpbs}}
%\newcommand{\kissat}{\textsc{kissat}}
%\newcommand{\Kissat}{\textsc{Kissat}}

% Code formatting
\newcommand{\showcomment}[1]{\texttt{/*} \textit{#1} \texttt{*/}}
\newcommand{\keyword}[1]{\textbf{#1}}
\newcommand{\keyif}{\keyword{if}}
\newcommand{\keyfor}{\keyword{for}}
\newcommand{\keyelse}{\keyword{else}}
\newcommand{\keyor}{\keyword{or}}
\newcommand{\keyreturn}{\keyword{return}}
\newcommand{\assign}{\ensuremath{\longleftarrow}}



\newcommand{\interp}{\sigma}
\newcommand{\interpset}[1]{\Sigma_{#1}}
\newcommand{\mcount}{\mu}

\newcommand{\ifarg}{\textbf{I}}
\newcommand{\thenarg}{\textbf{T}}
\newcommand{\elsearg}{\textbf{E}}
\newcommand{\depend}{{\it D}}

\newcommand{\subs}[2]{[#2/#1]}
\newcommand{\substrue}[1]{\subs{#1}{\tautology}}
\newcommand{\subsfalse}[1]{\subs{#1}{\false}}
\newcommand{\subsflip}[1]{\subs{#1}{\obar{#1}}}

\newcommand{\amax}{a_{{\rm max}}}


\title{Clausal Proofs for Pseudo-Boolean Reasoning \\ Experimental Data}

\titlerunning{Clausal Proofs from Pseudo-Boolean Reasoning}

\author{%
  Randal E. Bryant\inst{1}
  \and
  Armin Biere\inst{2}
  \and
  Marijn J. H. Heule\inst{1}
}

\institute{
  Carnegie Mellon University, Pittsburgh, PA, United States\\
  {\tt  \{Randy.Bryant, mheule\}@cs.cmu.edu}
  \and
  Albert-Ludwigs University, Freiburg, Germany\\
  {\tt  biere@cs.uni-freiburg.de}
}

\authorrunning{R. E. Bryant, A. Biere, and M. J. H. Heule}

\begin{document}

\maketitle

The attached graphs are extracted from the paper ``Clausal Proofs from
Pseudo-Boolean Reasoning.''  For each one, we show the complete data
presented in the paper, as well as reduced data generated by from the
artifact submission.  Although the reproduced results do not
demonstrate the full scale of the experimental results, they
demonstrate that our solvers far exceed the capabilities of \kissat{},
a state-of-the-art CDCL solver.


\newpage
\begin{figure}
A) Original results

\begin{tikzpicture}[scale = 1.0]
          \begin{axis}[mark options={scale=0.8},grid=both, grid style={black!10}, xmode=log, ymode=log, legend style={at={(1.00,0.37)}}, legend cell align={left},
                              x post scale=1.6, xlabel=$m$, xtick={2,4,8,16,32,64}, xticklabels={$2$,$4$,$8$,$16$,$32$,$64$},xmin=2,xmax=64,ymin=10000,ymax=100000000, title={Urquhart Clauses}]
            \input{original-data/urquhart-simon-kissat}                        
            \input{original-data/urquhart-simon-bucket}
            \input{original-data/urquhart-simon-equation}
            \input{original-data/urquhart-li-bucket}
            \input{original-data/urquhart-li-equation}
            \legend{
              \scriptsize \textsf{Simon, \kissat},
              \scriptsize \textsf{Simon, \pgbdd, Bucket Elimination},
              \scriptsize \textsf{Simon, \pgpbs, Mod-2 Equations},
              \scriptsize \textsf{Li, \pgbdd, Bucket Elimination},
              \scriptsize \textsf{Li, \pgpbs, Mod-2 Equations},
            }
          \end{axis}
\end{tikzpicture}

B) Reproduced results

\begin{tikzpicture}[scale = 1.0]
          \begin{axis}[mark options={scale=0.8},grid=both, grid style={black!10}, xmode=log, ymode=log, legend style={at={(1.00,0.37)}}, legend cell align={left},
                              x post scale=1.6, xlabel=$m$, xtick={2,4,8,16,32,64}, xticklabels={$2$,$4$,$8$,$16$,$32$,$64$},xmin=2,xmax=64,ymin=10000,ymax=100000000, title={Urquhart Clauses}]
%%            \input{reproduced-data/urquhart-simon-kissat}                        
            \input{reproduced-data/urquhart-simon-bucket}
            \input{reproduced-data/urquhart-simon-equation}
            \input{reproduced-data/urquhart-li-bucket}
            \input{reproduced-data/urquhart-li-equation}
            \legend{
%%              \scriptsize \textsf{Simon, \kissat},
              \scriptsize \textsf{Simon, \pgbdd, Bucket Elimination},
              \scriptsize \textsf{Simon, \pgpbs, Mod-2 Equations},
              \scriptsize \textsf{Li, \pgbdd, Bucket Elimination},
              \scriptsize \textsf{Li, \pgpbs, Mod-2 Equations},
            }
          \end{axis}
\end{tikzpicture}

\caption{Total number of clauses in proofs of two sets of Urquhart formulas.}
\label{fig:data:urquhart}
\end{figure}  

\newpage
\begin{figure}
A) Original results

\begin{tikzpicture}[scale = 1.0]
          \begin{axis}[mark options={scale=0.8},grid=both, grid style={black!10}, xmode=log, ymode=log, legend style={at={(1.00,0.30)}}, legend cell align={left},
                              x post scale=1.6, xlabel=$n$, xtick={4,8,16,32,64,128}, xticklabels={$4$,$8$,$16$,$32$,$64$,$128$},xmin=4,xmax=128,ymin=1000,ymax=100000000, title={Mutilated Chessboard Clauses}]
            \input{original-data/chess-kissat}
            \input{original-data/chess-equation-integer}
            \input{original-data/chess-column}
            \input{original-data/chess-equation-mod3}
            \legend{
              \scriptsize \textsf{\kissat},
              \scriptsize \textsf{\pgpbs, Integer Equations, Input Order},
              \scriptsize \textsf{\pgbdd, Column Scan, Input Order},
              \scriptsize \textsf{\pgpbs, Mod-3 Equations, Input Order},
            }
          \end{axis}
\end{tikzpicture}

B) Reproduced results

\begin{tikzpicture}[scale = 1.0]
          \begin{axis}[mark options={scale=0.8},grid=both, grid style={black!10}, xmode=log, ymode=log, legend style={at={(1.00,0.30)}}, legend cell align={left},
                              x post scale=1.6, xlabel=$n$, xtick={4,8,16,32,64,128}, xticklabels={$4$,$8$,$16$,$32$,$64$,$128$},xmin=4,xmax=128,ymin=1000,ymax=100000000, title={Mutilated Chessboard Clauses}]
%%            \input{reproduced-data/chess-kissat}
            \input{reproduced-data/chess-equation-integer}
            \input{reproduced-data/chess-column}
            \input{reproduced-data/chess-equation-mod3}
            \legend{
%%              \scriptsize \textsf{\kissat},
              \scriptsize \textsf{\pgpbs, Integer Equations, Input Order},
              \scriptsize \textsf{\pgbdd, Column Scan, Input Order},
              \scriptsize \textsf{\pgpbs, Mod-3 Equations, Input Order},
            }
          \end{axis}
\end{tikzpicture}

\caption{Total number of clauses in proofs of $n \times n$ mutilated
chess board problems.}% using different types of solvers.}
\label{fig:data:chess-baseline}
\end{figure}  

\newpage
\begin{figure}

A) Original results

\begin{tikzpicture}[scale = 1.0]
          \begin{axis}[mark options={scale=0.8},grid=both, grid style={black!10}, xmode=log, ymode=log, legend style={at={(0.99,0.30)}}, legend cell align={left},
                              x post scale=1.6, xlabel=$n$, xtick={4,8,16,32,64,128}, xticklabels={$4$,$8$,$16$,$32$,$64$,$128$},xmin=4,xmax=128,ymin=1000,ymax=100000000, title={Mutilated Chess Board/Torus Clauses}]
             \input{original-data/chess-board-column-randomorder}
            \input{original-data/chess-torus-column}          
            \input{original-data/chess-torus-equation-randomorder}
             \input{original-data/chess-board-equation-randomorder}
            \legend{
              \scriptsize \textsf{Board, \pgbdd, Column Scan, Random Order}, 
              \scriptsize \textsf{Torus, \pgbdd, Column Scan, Input Order}, 
              \scriptsize \textsf{Torus, \pgpbs, Autodetect, Random Order}, 
              \scriptsize \textsf{Board, \pgpbs, Autodetect, Random Order}, 
            }
          \end{axis}
\end{tikzpicture}

B) Reproduced results

\begin{tikzpicture}[scale = 1.0]
          \begin{axis}[mark options={scale=0.8},grid=both, grid style={black!10}, xmode=log, ymode=log, legend style={at={(0.99,0.30)}}, legend cell align={left},
                              x post scale=1.6, xlabel=$n$, xtick={4,8,16,32,64,128}, xticklabels={$4$,$8$,$16$,$32$,$64$,$128$},xmin=4,xmax=128,ymin=1000,ymax=100000000, title={Mutilated Chess Board/Torus Clauses}]
             \input{reproduced-data/chess-board-column-randomorder}
            \input{reproduced-data/chess-torus-column}          
            \input{reproduced-data/chess-torus-equation-randomorder}
             \input{reproduced-data/chess-board-equation-randomorder}
            \legend{
              \scriptsize \textsf{Board, \pgbdd, Column Scan, Random Order}, 
              \scriptsize \textsf{Torus, \pgbdd, Column Scan, Input Order}, 
              \scriptsize \textsf{Torus, \pgpbs, Autodetect, Random Order}, 
              \scriptsize \textsf{Board, \pgpbs, Autodetect, Random Order}, 
            }
          \end{axis}
\end{tikzpicture}


\caption{Stress Testing: Changing the topology and variable ordering for mutilated chess.
Autodetection enables the PB solver to use modulo-3 arithmetic.}
\label{fig:data:chess-stressed}
\end{figure}  

\newpage

\begin{figure}
A) Original results

\begin{tikzpicture}[scale = 1.0]
          \begin{axis}[mark options={scale=0.8},grid=both, grid style={black!10}, xmode=log, ymode=log, legend style={at={(1.00,0.42)}}, legend cell align={left},
                              x post scale=1.6, y post scale=1.15, xlabel=$n$, xtick={4,8,16,32,64,128, 256}, xticklabels={$4$,$8$,$16$,$32$,$64$,$128$},xmin=4,xmax=140,ymin=100,ymax=100000000, title={Pigeonhole Clauses}]
            \input{original-data/pigeon-direct-kissat}
            \input{original-data/pigeon-sinz-kissat}
            \input{original-data/pigeon-direct-linear-inputorder}
            \input{original-data/pigeon-direct-constraint-randomorder}
            \input{original-data/pigeon-sinz-equation-randomorder}
            \input{original-data/pigeon-sinz-column-inputorder}
            \input{original-data/pigeon-direct-cook}
 
            \legend{
              \scriptsize \textsf{Direct, \kissat},
              \scriptsize \textsf{Sinz, \kissat},
              \scriptsize \textsf{Direct, \pgbdd, Linear, Input Order},
              \scriptsize \textsf{Direct, \pgpbs, Constraints, Random Order\!\!\!},
              \scriptsize \textsf{Sinz, \pgpbs, Equations, Random Order},
              \scriptsize \textsf{Sinz, \pgbdd, Column Scan, Input Order},
              \scriptsize \textsf{Direct, Cook's Proof},
            }
          \end{axis}
\end{tikzpicture}

B) Reproduced results

\begin{tikzpicture}[scale = 1.0]
          \begin{axis}[mark options={scale=0.8},grid=both, grid style={black!10}, xmode=log, ymode=log, legend style={at={(1.00,0.42)}}, legend cell align={left},
                              x post scale=1.6, y post scale=1.15, xlabel=$n$, xtick={4,8,16,32,64,128, 256}, xticklabels={$4$,$8$,$16$,$32$,$64$,$128$},xmin=4,xmax=140,ymin=100,ymax=100000000, title={Pigeonhole Clauses}]
            \input{reproduced-data/pigeon-direct-linear-inputorder}
            \input{reproduced-data/pigeon-direct-constraint-randomorder}
            \input{reproduced-data/pigeon-sinz-equation-randomorder}
            \input{reproduced-data/pigeon-sinz-column-inputorder}
            \input{reproduced-data/pigeon-direct-cook}
 
            \legend{
              \scriptsize \textsf{Direct, \pgbdd, Linear, Input Order},
              \scriptsize \textsf{Direct, \pgpbs, Constraints, Random Order\!\!\!},
              \scriptsize \textsf{Sinz, \pgpbs, Equations, Random Order},
              \scriptsize \textsf{Sinz, \pgbdd, Column Scan, Input Order},
              \scriptsize \textsf{Direct, Cook's Proof},
            }
          \end{axis}
\end{tikzpicture}

\caption{Total number of clauses in proofs of pigeonhole problem for $n$ holes}
\label{fig:data:pigeonhole}
\end{figure}  

\end{document}
