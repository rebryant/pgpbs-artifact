\documentclass[runningheads]{llncs}

%%
%% Corresponding author:
%% Randal E. Bryant (Randy.Bryant@cs.cmu.edu)
%% 

%\usepackage{latexsym}
\usepackage{times}
\usepackage{tikz}
\usepackage{pgfplots}
\usepackage[many]{tcolorbox}


\definecolor{redorange}{rgb}{0.878431, 0.235294, 0.192157}
\definecolor{lightblue}{rgb}{0.95, 0.9, 0.99}
\definecolor{clearyellow}{rgb}{0.964706, 0.745098, 0}
\definecolor{clearorange}{rgb}{0.917647, 0.462745, 0}
\definecolor{mildgray}{rgb}{0.54902, 0.509804, 0.47451}
\definecolor{softblue}{rgb}{0.643137, 0.858824, 0.909804}
\definecolor{bluegray}{rgb}{0.141176, 0.313725, 0.603922}
\definecolor{lightgreen}{rgb}{0.9, 0.99, 0.9}
\definecolor{redpurple}{rgb}{0.835294, 0, 0.196078}
\definecolor{midblue}{rgb}{0, 0.592157, 0.662745}
\definecolor{clearpurple}{rgb}{0.67451, 0.0784314, 0.352941}
\definecolor{browngreen}{rgb}{0.333333, 0.313725, 0.145098}
\definecolor{darkestpurple}{rgb}{0.396078, 0.113725, 0.196078}
\definecolor{greypurple}{rgb}{0.294118, 0.219608, 0.298039}
\definecolor{darkturquoise}{rgb}{0, 0.239216, 0.298039}
\definecolor{darkbrown}{rgb}{0.305882, 0.211765, 0.160784}
\definecolor{midgreen}{rgb}{0.560784, 0.6, 0.243137}
\definecolor{darkred}{rgb}{0.576471, 0.152941, 0.172549}
\definecolor{darkpurple}{rgb}{0.313725, 0.027451, 0.470588}
\definecolor{darkestblue}{rgb}{0, 0.156863, 0.333333}
\definecolor{lightpurple}{rgb}{0.776471, 0.690196, 0.737255}
\definecolor{softgreen}{rgb}{0.733333, 0.772549, 0.572549}
\definecolor{medgreen}{rgb}{0.34, 0.65, 0.34}
\definecolor{offwhite}{rgb}{0.839216, 0.823529, 0.768627}


\newcommand{\pgbdd}{{\sffamily\scshape pgbdd}}
\newcommand{\Pgbdd}{{\sffamily\scshape  Pgbdd}}
\newcommand{\pgpbs}{{\sffamily\scshape  pgpbs}}
\newcommand{\Pgpbs}{{\sffamily\scshape  Pgpbs}}
\newcommand{\kissat}{{\sffamily\scshape kissat}}
\newcommand{\Kissat}{{\sffamily\scshape  Kissat}}
\newcommand{\lingeling}{{\sffamily\scshape lingeling}}
\newcommand{\Lingeling}{{\sffamily\scshape  Lingeling}}

%\newcommand{\pgbdd}{\textsc{pgbdd}}
%\newcommand{\Pgbdd}{\textsc{Pgbdd}}
%\newcommand{\pgpbs}{\textsc{pgpbs}}
%\newcommand{\Pgpbs}{\textsc{Pgpbs}}
%\newcommand{\kissat}{\textsc{kissat}}
%\newcommand{\Kissat}{\textsc{Kissat}}


\title{Clausal Proofs for Pseudo-Boolean Reasoning \\ Experimental Data}

\titlerunning{Clausal Proofs from Pseudo-Boolean Reasoning}

\author{%
  Randal E. Bryant\inst{1}
  \and
  Armin Biere\inst{2}
  \and
  Marijn J. H. Heule\inst{1}
}

\institute{
  Carnegie Mellon University, Pittsburgh, PA, United States\\
  {\tt  \{Randy.Bryant, mheule\}@cs.cmu.edu}
  \and
  Albert-Ludwigs University, Freiburg, Germany\\
  {\tt  biere@cs.uni-freiburg.de}
}

\authorrunning{R. E. Bryant, A. Biere, and M. J. H. Heule}

\begin{document}

\maketitle

The attached graphs are extracted from the experimental results
section of the paper ``Clausal Proofs from Pseudo-Boolean Reasoning.''
For each one, we show the complete data presented in the paper, as
well as reduced data generated from the artifact.
Although the reproduced results do not demonstrate the full scale of
the experimental results, they demonstrate that our solvers far exceed
the capabilities of \kissat{}, a state-of-the-art CDCL solver.

There are three classes of problems considered.  All of the formulas
are unsatisfiable, and so the task is to generate a proof of
unsatisfiability.

In particular, here is the significance of the reproduced results for
each of the four figures in the paper:
\begin{description}
\item[Fig.~4] These are for a benchmark based on a family of formulas
  devised by Urquhart, designed to require resolution proofs of
  exponential size.  There are two versions of these formulas, due to
  Simon and to Li, with Li's versions being more challenging.  The
  formulas scale quadratically with the parameter $m$, with $m \geq
  3$.  \Kissat{} can handle some instances of Simon's benchmarks for
  $m \in \{3,4\}$, but none of Li's benchmarks.  \Pgpbs{} generates a
  proof of unsatisfiability for $m=12$ in under 20 seconds.

\item[Fig.~5] These are for the mutilated chessboard problem,
  concerning the tiling of an $n \times n$ chessboard with two corners
  removed.  As the figure shows, \kissat{} scales exponentially with
  $n$.  Our earlier solver \pgbdd{} scales as $O(n^3)$ with careful
  guidance.  The new solver \pgpbs{} can achieve this scaling with
  full automation. Both programs  generate proofs for up to
  $n=32$ in under 15 seconds.

\item[Fig.~6] These further demonstrate the advantages of \pgpbs{}
  over \pgbdd{}.  \Pgbdd{} has exponential scaling when either 1) the
  input are ordered according to a random permutation, or 2) the chess
  board is converted to a torus by having the ends wrap around.
  \Pgpbs{} maintains the $O(n^3)$ scaling even with these variations.
  The ability to work efficiently with arbitrary variable orderings is
  unusual for BDD-based applications.  The reproduced results go up to
  $n=32$ for the variant problems.

\item[Fig.~7] These are for the pigeonhole problem, concerning the
  impossible task of assigning of $n+1$ pigeons to $n$ holes such that
  each hole contains at most one pigeon.  The at-most-one constraints
  of pigeons to holes can be encoded either {\em directly} requiring
  $O(n^2)$ clauses per constraint, or via an encoding due to Sinz,
  requiring $O(n)$ clauses.  \Kissat{} scales exponentially for either
  encoding, as does \pgbdd{} for a direct encoding.  \Pgbdd{} can
  achieve $O(n^3)$ scaling on the Sinz encoding with careful guidance
  from the user.  \Pgpbs{} can achieve $O(n^5)$ encoding with either
  encoding and with no guidance, or $O(n^3)$ performance when the
  at-most-one constraints are tightened to exactly-one constraints.
  Finally, Cook showed an inductive approach to proving the
  unsatisfiability of the problem that scales as $O(n^4)$, but with a
  very low constant factor.  The reproduced results range up to $n=16$
  for versions scaling as $O(n^5)$ and up to $n=32$ for those achieving
  $O(n^3)$ or $O(n^4)$ scaling.
 
\end{description}


\setcounter{figure}{3}

\newpage
\begin{figure}
A) Original results

\begin{tikzpicture}[scale = 1.0]
          \begin{axis}[mark options={scale=0.8},grid=both, grid style={black!10}, xmode=log, ymode=log, legend style={at={(1.00,0.37)}}, legend cell align={left},
                              x post scale=1.6, xlabel=$m$, xtick={2,4,8,16,32,64}, xticklabels={$2$,$4$,$8$,$16$,$32$,$64$},xmin=2,xmax=64,ymin=10000,ymax=100000000, title={Urquhart Clauses}]
            \input{original-data/urquhart-simon-kissat}                        
            \input{original-data/urquhart-simon-bucket}
            \input{original-data/urquhart-simon-equation}
            \input{original-data/urquhart-li-bucket}
            \input{original-data/urquhart-li-equation}
            \legend{
              \scriptsize \textsf{Simon, \kissat},
              \scriptsize \textsf{Simon, \pgbdd, Bucket Elimination},
              \scriptsize \textsf{Simon, \pgpbs, Mod-2 Equations},
              \scriptsize \textsf{Li, \pgbdd, Bucket Elimination},
              \scriptsize \textsf{Li, \pgpbs, Mod-2 Equations},
            }
          \end{axis}
\end{tikzpicture}

B) Reproduced results

\begin{tikzpicture}[scale = 1.0]
          \begin{axis}[mark options={scale=0.8},grid=both, grid style={black!10}, xmode=log, ymode=log, legend style={at={(1.00,0.37)}}, legend cell align={left},
                              x post scale=1.6, xlabel=$m$, xtick={2,4,8,16,32,64}, xticklabels={$2$,$4$,$8$,$16$,$32$,$64$},xmin=2,xmax=64,ymin=10000,ymax=100000000, title={Urquhart Clauses}]
%%            \input{reproduced-data/urquhart-simon-kissat}                        
            \input{reproduced-data/urquhart-simon-bucket}
            \input{reproduced-data/urquhart-simon-equation}
            \input{reproduced-data/urquhart-li-bucket}
            \input{reproduced-data/urquhart-li-equation}
            \legend{
%%              \scriptsize \textsf{Simon, \kissat},
              \scriptsize \textsf{Simon, \pgbdd, Bucket Elimination},
              \scriptsize \textsf{Simon, \pgpbs, Mod-2 Equations},
              \scriptsize \textsf{Li, \pgbdd, Bucket Elimination},
              \scriptsize \textsf{Li, \pgpbs, Mod-2 Equations},
            }
          \end{axis}
\end{tikzpicture}

\caption{Total number of clauses in proofs of two sets of Urquhart formulas.}
\label{fig:data:urquhart}
\end{figure}  

\newpage
\begin{figure}
A) Original results

\begin{tikzpicture}[scale = 1.0]
          \begin{axis}[mark options={scale=0.8},grid=both, grid style={black!10}, xmode=log, ymode=log, legend style={at={(1.00,0.30)}}, legend cell align={left},
                              x post scale=1.6, xlabel=$n$, xtick={4,8,16,32,64,128}, xticklabels={$4$,$8$,$16$,$32$,$64$,$128$},xmin=4,xmax=128,ymin=1000,ymax=100000000, title={Mutilated Chessboard Clauses}]
            \input{original-data/chess-kissat}
            \input{original-data/chess-equation-integer}
            \input{original-data/chess-column}
            \input{original-data/chess-equation-mod3}
            \legend{
              \scriptsize \textsf{\kissat},
              \scriptsize \textsf{\pgpbs, Integer Equations, Input Order},
              \scriptsize \textsf{\pgbdd, Column Scan, Input Order},
              \scriptsize \textsf{\pgpbs, Mod-3 Equations, Input Order},
            }
          \end{axis}
\end{tikzpicture}

B) Reproduced results

\begin{tikzpicture}[scale = 1.0]
          \begin{axis}[mark options={scale=0.8},grid=both, grid style={black!10}, xmode=log, ymode=log, legend style={at={(1.00,0.30)}}, legend cell align={left},
                              x post scale=1.6, xlabel=$n$, xtick={4,8,16,32,64,128}, xticklabels={$4$,$8$,$16$,$32$,$64$,$128$},xmin=4,xmax=128,ymin=1000,ymax=100000000, title={Mutilated Chessboard Clauses}]
%%            \input{reproduced-data/chess-kissat}
            \input{reproduced-data/chess-equation-integer}
            \input{reproduced-data/chess-column}
            \input{reproduced-data/chess-equation-mod3}
            \legend{
%%              \scriptsize \textsf{\kissat},
              \scriptsize \textsf{\pgpbs, Integer Equations, Input Order},
              \scriptsize \textsf{\pgbdd, Column Scan, Input Order},
              \scriptsize \textsf{\pgpbs, Mod-3 Equations, Input Order},
            }
          \end{axis}
\end{tikzpicture}

\caption{Total number of clauses in proofs of $n \times n$ mutilated
chess board problems.}% using different types of solvers.}
\label{fig:data:chess-baseline}
\end{figure}  

\newpage
\begin{figure}

A) Original results

\begin{tikzpicture}[scale = 1.0]
          \begin{axis}[mark options={scale=0.8},grid=both, grid style={black!10}, xmode=log, ymode=log, legend style={at={(0.99,0.30)}}, legend cell align={left},
                              x post scale=1.6, xlabel=$n$, xtick={4,8,16,32,64,128}, xticklabels={$4$,$8$,$16$,$32$,$64$,$128$},xmin=4,xmax=128,ymin=1000,ymax=100000000, title={Mutilated Chess Board/Torus Clauses}]
             \input{original-data/chess-board-column-randomorder}
            \input{original-data/chess-torus-column}          
            \input{original-data/chess-torus-equation-randomorder}
             \input{original-data/chess-board-equation-randomorder}
            \legend{
              \scriptsize \textsf{Board, \pgbdd, Column Scan, Random Order}, 
              \scriptsize \textsf{Torus, \pgbdd, Column Scan, Input Order}, 
              \scriptsize \textsf{Torus, \pgpbs, Autodetect, Random Order}, 
              \scriptsize \textsf{Board, \pgpbs, Autodetect, Random Order}, 
            }
          \end{axis}
\end{tikzpicture}

B) Reproduced results

\begin{tikzpicture}[scale = 1.0]
          \begin{axis}[mark options={scale=0.8},grid=both, grid style={black!10}, xmode=log, ymode=log, legend style={at={(0.99,0.30)}}, legend cell align={left},
                              x post scale=1.6, xlabel=$n$, xtick={4,8,16,32,64,128}, xticklabels={$4$,$8$,$16$,$32$,$64$,$128$},xmin=4,xmax=128,ymin=1000,ymax=100000000, title={Mutilated Chess Board/Torus Clauses}]
             \input{reproduced-data/chess-board-column-randomorder}
            \input{reproduced-data/chess-torus-column}          
            \input{reproduced-data/chess-torus-equation-randomorder}
             \input{reproduced-data/chess-board-equation-randomorder}
            \legend{
              \scriptsize \textsf{Board, \pgbdd, Column Scan, Random Order}, 
              \scriptsize \textsf{Torus, \pgbdd, Column Scan, Input Order}, 
              \scriptsize \textsf{Torus, \pgpbs, Autodetect, Random Order}, 
              \scriptsize \textsf{Board, \pgpbs, Autodetect, Random Order}, 
            }
          \end{axis}
\end{tikzpicture}


\caption{Stress Testing: Changing the topology and variable ordering for mutilated chess.
Autodetection enables the PB solver to use modulo-3 arithmetic.}
\label{fig:data:chess-stressed}
\end{figure}  

\newpage

\begin{figure}
A) Original results

\begin{tikzpicture}[scale = 1.0]
          \begin{axis}[mark options={scale=0.8},grid=both, grid style={black!10}, xmode=log, ymode=log, legend style={at={(1.00,0.42)}}, legend cell align={left},
                              x post scale=1.6, y post scale=1.15, xlabel=$n$, xtick={4,8,16,32,64,128, 256}, xticklabels={$4$,$8$,$16$,$32$,$64$,$128$},xmin=4,xmax=140,ymin=100,ymax=100000000, title={Pigeonhole Clauses}]
            \input{original-data/pigeon-direct-kissat}
            \input{original-data/pigeon-sinz-kissat}
            \input{original-data/pigeon-direct-linear-inputorder}
            \input{original-data/pigeon-direct-constraint-randomorder}
            \input{original-data/pigeon-sinz-equation-randomorder}
            \input{original-data/pigeon-sinz-column-inputorder}
            \input{original-data/pigeon-direct-cook}
 
            \legend{
              \scriptsize \textsf{Direct, \kissat},
              \scriptsize \textsf{Sinz, \kissat},
              \scriptsize \textsf{Direct, \pgbdd, Linear, Input Order},
              \scriptsize \textsf{Direct, \pgpbs, Constraints, Random Order\!\!\!},
              \scriptsize \textsf{Sinz, \pgpbs, Equations, Random Order},
              \scriptsize \textsf{Sinz, \pgbdd, Column Scan, Input Order},
              \scriptsize \textsf{Direct, Cook's Proof},
            }
          \end{axis}
\end{tikzpicture}

B) Reproduced results

\begin{tikzpicture}[scale = 1.0]
          \begin{axis}[mark options={scale=0.8},grid=both, grid style={black!10}, xmode=log, ymode=log, legend style={at={(1.00,0.42)}}, legend cell align={left},
                              x post scale=1.6, y post scale=1.15, xlabel=$n$, xtick={4,8,16,32,64,128, 256}, xticklabels={$4$,$8$,$16$,$32$,$64$,$128$},xmin=4,xmax=140,ymin=100,ymax=100000000, title={Pigeonhole Clauses}]
            \input{reproduced-data/pigeon-direct-linear-inputorder}
            \input{reproduced-data/pigeon-direct-constraint-randomorder}
            \input{reproduced-data/pigeon-sinz-equation-randomorder}
            \input{reproduced-data/pigeon-sinz-column-inputorder}
            \input{reproduced-data/pigeon-direct-cook}
 
            \legend{
              \scriptsize \textsf{Direct, \pgbdd, Linear, Input Order},
              \scriptsize \textsf{Direct, \pgpbs, Constraints, Random Order\!\!\!},
              \scriptsize \textsf{Sinz, \pgpbs, Equations, Random Order},
              \scriptsize \textsf{Sinz, \pgbdd, Column Scan, Input Order},
              \scriptsize \textsf{Direct, Cook's Proof},
            }
          \end{axis}
\end{tikzpicture}

\caption{Total number of clauses in proofs of pigeonhole problem for $n$ holes}
\label{fig:data:pigeonhole}
\end{figure}  

\end{document}
